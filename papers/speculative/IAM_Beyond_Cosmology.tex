{\rtf1\ansi\ansicpg1252\cocoartf2867
\cocoatextscaling0\cocoaplatform0{\fonttbl\f0\fswiss\fcharset0 Helvetica;}
{\colortbl;\red255\green255\blue255;}
{\*\expandedcolortbl;;}
\margl1440\margr1440\vieww11520\viewh8400\viewkind0
\pard\tx720\tx1440\tx2160\tx2880\tx3600\tx4320\tx5040\tx5760\tx6480\tx7200\tx7920\tx8640\pardirnatural\partightenfactor0

\f0\fs24 \cf0 \\documentclass[11pt]\{article\}\
\\usepackage[margin=1in]\{geometry\}\
\\usepackage\{amsmath\}\
\\usepackage\{amssymb\}\
\\usepackage\{graphicx\}\
\\usepackage\{hyperref\}\
\\usepackage\{cleveref\}\
\\usepackage\{enumitem\}\
\
\\title\{\\textbf\{Beyond Cosmology: Implications of the Informational Actualization Model Across Physical Scales\}\\\\\
\\vspace\{0.3cm\}\
\\large A Speculative Framework for Quantum Computing, Thermodynamics, and Computation\}\
\
\\author\{Heath W. Mahaffey\\\\\
\\small Independent Researcher\\\\\
\\small Entiat, WA 98822, USA\\\\\
\\small \\texttt\{hmaffeyges@gmail.com\}\\\\\
\\vspace\{0.2cm\}\
\\small \\textit\{With collaborative assistance from AI (Claude, Anthropic)\}\}\
\
\\date\{February 2026\}\
\
\\begin\{document\}\
\
\\maketitle\
\
\\begin\{abstract\}\
The Informational Actualization Model (IAM), originally developed to resolve the Hubble tension in cosmology, proposes that cosmic expansion emerges from irreversible informational actualization on the apparent horizon. If this principle is fundamental rather than emergent, it should apply across all physical scales---from quantum mechanics to thermodynamics to computation. This speculative paper explores the potential implications of IAM beyond cosmology, identifying testable predictions, technological opportunities, and paradigm shifts that may follow if the cosmological framework is validated. We propose that actualization---the irreversible transition from potential to actual states---represents a fundamental physical principle with profound consequences for quantum computing limits, thermodynamic efficiency bounds, measurement theory, and computational architecture. This work serves as a timestamp for future exploration and collaboration, developed through human-AI partnership in pattern recognition and cross-domain synthesis.\
\\end\{abstract\}\
\
\\section\{Introduction\}\
\
The Informational Actualization Model (IAM) was developed to address the $>5\\sigma$ Hubble tension between early-universe and late-time measurements of the expansion rate \\cite\{Mahaffey2026\}. By coupling expansion dynamics to the linear growth factor $D(z)$ through horizon thermodynamics, IAM achieves superior statistical fit ($\\Delta\\chi^2 = 59.58$, $5.7\\sigma$ evidence) compared to $\\Lambda$CDM when tested against Planck, SH0ES, JWST, and DESI data.\
\
However, the implications of IAM extend far beyond resolving observational tensions in cosmology. If informational actualization is a \\textit\{fundamental\} rather than \\textit\{emergent\} feature of physical law, the same principle should manifest across all scales---from quantum wavefunctions to macroscopic thermodynamic systems to computational architectures.\
\
This paper explores these broader implications in a speculative but systematic framework. We identify:\
\\begin\{enumerate\}[label=(\\roman*)]\
    \\item Fundamental limits IAM predicts for quantum computing and reversible computation\
    \\item New technological opportunities in actualization-based computing and thermodynamic engines\
    \\item Resolutions to long-standing foundational puzzles (measurement problem, arrow of time)\
    \\item A roadmap for experimental validation across domains\
    \\item Potential paradigm shifts in physics, computer science, and engineering\
\\end\{enumerate\}\
\
This work is published as a timestamp and invitation for collaboration. It represents cross-domain pattern recognition developed through human-AI partnership, demonstrating a new methodology for exploratory theoretical physics.\
\
\\section\{Core Principle: Actualization as Fundamental\}\
\
\\subsection\{Definition\}\
\
\\textbf\{Actualization\} is the irreversible process by which potential states transition to definite, information-bearing states. In philosophical terms (Aristotelian-Thomistic metaphysics), this is the transition from \\textit\{potentia\} to \\textit\{actus\}.\
\
In physical terms:\
\\begin\{itemize\}\
    \\item \\textbf\{Quantum scale:\} Wavefunction collapse; superposition $\\rightarrow$ definite eigenstate\
    \\item \\textbf\{Thermodynamic scale:\} Entropy increase; microstates $\\rightarrow$ macroscopic observables\
    \\item \\textbf\{Cosmological scale:\} Structure formation; homogeneous plasma $\\rightarrow$ galaxies, stars, planets\
\\end\{itemize\}\
\
\\subsection\{Irreversibility\}\
\
A key claim of IAM is that actualization is \\textit\{thermodynamically irreversible\}. Once information is actualized (encoded in definite states), reversal requires:\
\\begin\{equation\}\
\\Delta E_\{\\text\{reversal\}\} \\geq k_B T \\ln 2 \\times I_\{\\text\{actualized\}\}\
\\end\{equation\}\
where $I_\{\\text\{actualized\}\}$ is the information (in bits) that has been encoded in the system, and the bound follows from Landauer's principle \\cite\{Landauer1961\}.\
\
This implies:\
\\begin\{itemize\}\
    \\item True reversibility is thermodynamically costly, not free\
    \\item The arrow of time emerges from cumulative actualization\
    \\item Systems naturally evolve toward more actualized (higher information) states\
\\end\{itemize\}\
\
\\subsection\{Hypothesis: Universal Applicability\}\
\
\\textbf\{Central Hypothesis:\} If IAM correctly describes cosmological expansion, the principle of irreversible informational actualization is fundamental and applies universally across physical scales.\
\
\\textbf\{Testable Consequence:\} Predictions derived from actualization dynamics at quantum and thermodynamic scales should be experimentally verifiable, independent of cosmological validation.\
\
\\section\{Implications for Quantum Computing\}\
\
\\subsection\{Current Paradigm Assumptions\}\
\
Quantum computing is predicated on several key assumptions:\
\\begin\{enumerate\}\
    \\item \\textbf\{Reversibility:\} Unitary quantum gates are perfectly reversible ($U^\\dagger U = I$)\
    \\item \\textbf\{Coherence maintenance:\} Decoherence is external noise to be isolated against\
    \\item \\textbf\{Scalability:\} With sufficient qubits and error correction, arbitrary quantum computations are feasible\
\\end\{enumerate\}\
\
\\subsection\{IAM Predictions: Fundamental Limits\}\
\
If actualization is fundamental, quantum computing faces intrinsic constraints:\
\
\\subsubsection\{Limit 1: Quasi-Reversibility\}\
\
\\textbf\{Prediction:\} Quantum gates are not perfectly reversible. Each gate operation incurs a small but nonzero thermodynamic cost due to actualization of intermediate information states.\
\
\\textbf\{Mathematical form:\}\
\\begin\{equation\}\
\\Delta E_\{\\text\{gate\}\} \\geq \\epsilon \\cdot k_B T \\ln 2\
\\end\{equation\}\
where $\\epsilon \\ll 1$ but $\\epsilon > 0$ represents the actualization fraction per gate operation.\
\
\\textbf\{Experimental test:\}\
\\begin\{itemize\}\
    \\item Measure energy dissipation in high-fidelity quantum gates (superconducting, trapped ion)\
    \\item Run gate forward, then attempt perfect reversal\
    \\item Detect small thermodynamic signature distinguishing from truly reversible process\
\\end\{itemize\}\
\
\\subsubsection\{Limit 2: Decoherence as Intrinsic Actualization\}\
\
\\textbf\{Prediction:\} Decoherence is not merely environmental noise but the system's intrinsic tendency to actualize into definite states. Isolation reduces decoherence rate but cannot eliminate it.\
\
\\textbf\{Scaling law:\}\
\\begin\{equation\}\
\\Gamma_\{\\text\{decoherence\}\} = \\Gamma_0 \\cdot N \\cdot f(T)\
\\end\{equation\}\
where $\\Gamma_0$ is an intrinsic actualization constant, $N$ is the number of qubits, and $f(T)$ is a temperature-dependent function.\
\
\\textbf\{Consequence:\} As $N$ increases, decoherence becomes increasingly difficult to suppress, imposing practical upper limits on quantum computer size.\
\
\\textbf\{Experimental test:\}\
\\begin\{itemize\}\
    \\item Measure decoherence time $T_2$ for systems of varying qubit number $N$\
    \\item Plot $1/T_2$ vs. $N$; IAM predicts linear scaling\
    \\item Compare to environmental decoherence models (predict logarithmic or sublinear scaling)\
\\end\{itemize\}\
\
\\subsubsection\{Limit 3: Error Correction Crossover\}\
\
\\textbf\{Prediction:\} Quantum error correction (QEC) fights actualization. Beyond a critical system size, the thermodynamic cost of error correction exceeds the computational advantage gained.\
\
\\textbf\{Crossover condition:\}\
\\begin\{equation\}\
N_\{\\text\{max\}\} \\sim \\frac\{E_\{\\text\{computation\}\}\}\{\\Gamma_0 \\cdot k_B T \\ln 2\}\
\\end\{equation\}\
\
\\textbf\{Implication:\} There exists a fundamental upper bound on practical quantum computer size, determined by actualization dynamics rather than engineering challenges alone.\
\
\\textbf\{Experimental/theoretical test:\}\
\\begin\{itemize\}\
    \\item Calculate total energy budget (computation + error correction) for large-scale QC architectures\
    \\item Identify crossover point where QEC overhead dominates\
    \\item Compare to IAM-predicted $N_\{\\text\{max\}\}$\
\\end\{itemize\}\
\
\\subsection\{Technological Pivot: Actualization-Based Computing\}\
\
Rather than fighting actualization, \\textbf\{exploit it\}.\
\
\\subsubsection\{Concept: Guided Actualization Computing\}\
\
\\textbf\{Principle:\} Initialize system in superposition, then guide actualization along energetically favorable pathways to solve computational problems.\
\
\\textbf\{Analogy:\} \
\\begin\{itemize\}\
    \\item Classical computing: bits already actualized (deterministic)\
    \\item Quantum computing: fight actualization to maintain coherence\
    \\item Actualization computing: \\textit\{surf\} the actualization process\
\\end\{itemize\}\
\
\\textbf\{Potential advantages:\}\
\\begin\{itemize\}\
    \\item No need for extreme isolation (work with environment, not against it)\
    \\item Harvest information from actualization dynamics itself\
    \\item Exploit thermodynamic gradients to drive computation\
\\end\{itemize\}\
\
\\textbf\{Applications:\}\
\\begin\{itemize\}\
    \\item Optimization problems (let system actualize to ground state)\
    \\item Pattern recognition (actualization reveals structure)\
    \\item Thermodynamic computing (use heat flow to drive actualization-based logic)\
\\end\{itemize\}\
\
\\section\{Implications for Thermodynamics and Energy\}\
\
\\subsection\{Landauer's Principle and Actualization\}\
\
Landauer's principle states that erasing one bit of information dissipates at least $k_B T \\ln 2$ of energy \\cite\{Landauer1961\}. IAM generalizes this:\
\
\\textbf\{Actualization dissipates energy:\}\
\\begin\{equation\}\
\\Delta E_\{\\text\{actualization\}\} \\geq k_B T \\ln 2 \\times \\Delta I\
\\end\{equation\}\
where $\\Delta I$ is the information actualized (measured in bits).\
\
\\subsection\{Thermodynamic Engines Exploiting Actualization\}\
\
\\subsubsection\{Actualization Heat Engines\}\
\
\\textbf\{Concept:\} Traditional heat engines exploit temperature gradients. Actualization engines exploit \\textit\{information gradients\}---moving systems from low-information (potential) to high-information (actual) states while harvesting energy.\
\
\\textbf\{Mechanism:\}\
\\begin\{enumerate\}\
    \\item Prepare system in low-entropy, high-potential state\
    \\item Allow controlled actualization (e.g., measurement, decoherence)\
    \\item Extract work from the actualization process\
    \\item Reset to initial state (requires energy input, but net gain possible)\
\\end\{enumerate\}\
\
\\textbf\{Theoretical efficiency bound:\}\
\\begin\{equation\}\
\\eta_\{\\text\{actualization\}\} \\leq 1 - \\frac\{T_\{\\text\{sink\}\}\}\{T_\{\\text\{source\}\}\} \\times \\frac\{\\Delta S_\{\\text\{thermal\}\}\}\{\\Delta I_\{\\text\{actualization\}\}\}\
\\end\{equation\}\
\
\\textbf\{Potential applications:\}\
\\begin\{itemize\}\
    \\item Micro-scale energy harvesting from environmental fluctuations\
    \\item Quantum thermal engines\
    \\item Information-to-energy conversion devices\
\\end\{itemize\}\
\
\\subsubsection\{Maxwell's Demon Revisited\}\
\
IAM provides a resolution to Maxwell's demon paradox:\
\
\\textbf\{Traditional problem:\} Demon sorts molecules, decreasing entropy without work input\'97violates second law.\
\
\\textbf\{IAM resolution:\} The demon must \\textit\{measure\} (actualize information about) molecular positions. This actualization has thermodynamic cost $\\geq k_B T \\ln 2$ per bit, restoring second law consistency.\
\
\\textbf\{Insight:\} Information acquisition is not free; it is actualization, which is thermodynamically costly.\
\
\\subsection\{New Material and Device Concepts\}\
\
\\subsubsection\{Actualization-Responsive Materials\}\
\
Materials engineered to undergo phase transitions or property changes in response to information actualization (e.g., measurement, observation).\
\
\\textbf\{Examples:\}\
\\begin\{itemize\}\
    \\item Quantum materials whose conductivity changes upon measurement\
    \\item Self-organizing systems driven by actualization dynamics\
    \\item ``Smart'' materials that respond to information flow\
\\end\{itemize\}\
\
\\subsubsection\{Information Batteries\}\
\
Devices that store energy in \\textit\{unactualized\} (potential) states and release it upon actualization.\
\
\\textbf\{Mechanism:\}\
\\begin\{itemize\}\
    \\item Charge: Prepare system in quantum superposition or metastable high-entropy state\
    \\item Store: Maintain isolation to prevent actualization\
    \\item Discharge: Trigger actualization (measurement, decoherence), harvest released energy\
\\end\{itemize\}\
\
\\textbf\{Advantages:\}\
\\begin\{itemize\}\
    \\item High energy density (information storage scales with Hilbert space dimension)\
    \\item Rapid discharge (actualization is fast)\
    \\item Tunable output (control actualization rate)\
\\end\{itemize\}\
\
\\section\{Resolution of Foundational Puzzles\}\
\
\\subsection\{The Quantum Measurement Problem\}\
\
\\textbf\{Traditional problem:\} What causes wavefunction collapse? Why does measurement yield definite outcomes from superposition?\
\
\\textbf\{IAM resolution:\} Measurement \\textit\{is\} actualization. The wavefunction collapse is the physical process of information transitioning from potential (superposition) to actual (definite eigenstate). This process is:\
\\begin\{itemize\}\
    \\item Irreversible (thermodynamically)\
    \\item Universal (applies to all quantum systems)\
    \\item Grounded in the same principle governing cosmic expansion\
\\end\{itemize\}\
\
\\textbf\{Prediction:\} Measurement energy cost should scale with information actualized:\
\\begin\{equation\}\
\\Delta E_\{\\text\{measurement\}\} \\geq k_B T \\ln(d)\
\\end\{equation\}\
where $d$ is the dimensionality of the measured observable's eigenspace.\
\
\\textbf\{Experimental test:\}\
\\begin\{itemize\}\
    \\item Measure thermodynamic signature of wavefunction collapse\
    \\item Vary measurement strength (weak vs. projective)\
    \\item Test energy scaling with information gain\
\\end\{itemize\}\
\
\\subsection\{The Arrow of Time\}\
\
\\textbf\{Traditional problem:\} Why does time have a direction? Fundamental laws are time-symmetric, yet we experience irreversible flow.\
\
\\textbf\{IAM resolution:\} The arrow of time \\textit\{is\} the direction of cumulative actualization. As the universe evolves:\
\\begin\{itemize\}\
    \\item More information is actualized (structure forms)\
    \\item Entropy increases (consistent with second law)\
    \\item The cosmos becomes progressively more ``actual'' and less ``potential''\
\\end\{itemize\}\
\
Time is the measure of actualization. Reversing time would require \\textit\{unactualizing\} information, which violates thermodynamic bounds.\
\
\\textbf\{Connection to cosmology:\} Cosmic expansion itself is the ultimate manifestation of time's arrow\'97the universe actualizing structure from the initial low-entropy state.\
\
\\subsection\{Entropy and Information\}\
\
\\textbf\{IAM unifies thermodynamic and informational entropy:\}\
\
\\begin\{equation\}\
S_\{\\text\{thermodynamic\}\} = k_B \\ln \\Omega \\quad \\Leftrightarrow \\quad S_\{\\text\{information\}\} = -\\sum_i p_i \\ln p_i\
\\end\{equation\}\
\
Both are measures of actualization:\
\\begin\{itemize\}\
    \\item Low entropy = high potentiality (many possible futures)\
    \\item High entropy = high actuality (definite, information-rich state)\
    \\item Entropy increase = ongoing actualization\
\\end\{itemize\}\
\
\\section\{Cross-Domain Predictions and Tests\}\
\
\\subsection\{Summary of Testable Predictions\}\
\
\\begin\{table\}[h!]\
\\centering\
\\small\
\\begin\{tabular\}\{|p\{3cm\}|p\{5.5cm\}|p\{5.5cm\}|\}\
\\hline\
\\textbf\{Domain\} & \\textbf\{Prediction\} & \\textbf\{Experimental Test\} \\\\\
\\hline\
Quantum Computing & Small but nonzero dissipation in reversible gates & Measure thermodynamic cost of gate reversal in superconducting qubits \\\\\
\\hline\
Quantum Computing & Decoherence scales linearly with qubit number & Measure $T_2$ vs. $N$ for varying system sizes \\\\\
\\hline\
Quantum Computing & QEC crossover at critical system size & Calculate energy budget for large-scale QC; identify $N_\{\\text\{max\}\}$ \\\\\
\\hline\
Measurement Theory & Energy cost scales with information gain & Measure thermodynamic signature of wavefunction collapse \\\\\
\\hline\
Thermodynamics & Actualization-driven heat engines possible & Build prototype actualization engine; measure efficiency \\\\\
\\hline\
Cosmology (baseline) & IAM fits BAO + $f\\sigma_8$ data better than $\\Lambda$CDM & Already tested: $\\Delta\\chi^2 = 59.58$ (\\cite\{Mahaffey2026\}) \\\\\
\\hline\
\\end\{tabular\}\
\\caption\{Summary of testable predictions across physical domains.\}\
\\label\{tab:predictions\}\
\\end\{table\}\
\
\\subsection\{Experimental Roadmap\}\
\
\\subsubsection\{Near-Term (1--3 years)\}\
\\begin\{itemize\}\
    \\item Quantum gate reversibility tests (IBM, Google quantum labs)\
    \\item Decoherence scaling measurements (academic trapped-ion labs)\
    \\item Measurement thermodynamics (precision calorimetry on quantum systems)\
\\end\{itemize\}\
\
\\subsubsection\{Medium-Term (3--7 years)\}\
\\begin\{itemize\}\
    \\item Prototype actualization-based computing devices\
    \\item Actualization heat engine demonstrations\
    \\item Large-scale QC energy budget analysis\
\\end\{itemize\}\
\
\\subsubsection\{Long-Term (7--15 years)\}\
\\begin\{itemize\}\
    \\item Commercialization of actualization computing architectures\
    \\item Integration into thermodynamic power generation\
    \\item Paradigm shift in quantum foundations and computation theory\
\\end\{itemize\}\
\
\\section\{Paradigm Shifts and Strategic Opportunities\}\
\
\\subsection\{Fields Affected if IAM is Validated\}\
\
\\subsubsection\{Quantum Computing Industry\}\
\\textbf\{Impact:\} Recognition of fundamental limits may redirect investment toward:\
\\begin\{itemize\}\
    \\item Hybrid quantum-classical architectures optimized for actualization dynamics\
    \\item Actualization computing as alternative paradigm\
    \\item Targeted applications where decoherence is acceptable or exploitable\
\\end\{itemize\}\
\
\\textbf\{Opportunity:\} Early adopters of actualization-based designs gain competitive advantage.\
\
\\subsubsection\{Thermodynamics and Energy\}\
\\textbf\{Impact:\} New class of thermodynamic devices exploiting information-energy conversion.\
\
\\textbf\{Opportunity:\} Patents on actualization engines, information batteries, Maxwell's demon-inspired devices.\
\
\\subsubsection\{Foundations of Physics\}\
\\textbf\{Impact:\} Resolution of measurement problem and arrow of time via actualization principle.\
\
\\textbf\{Opportunity:\} Unified framework spanning quantum mechanics, thermodynamics, and cosmology\'97comparable to how relativity unified space and time.\
\
\\subsubsection\{Computer Science and Computation Theory\}\
\\textbf\{Impact:\} New computational complexity classes based on actualization dynamics; rethinking of reversible vs. irreversible computation.\
\
\\textbf\{Opportunity:\} Novel algorithms exploiting guided actualization.\
\
\\subsubsection\{Philosophy of Science\}\
\\textbf\{Impact:\} Rehabilitation of teleology and formal causation in physics (actualization as goal-directed process toward definite states).\
\
\\textbf\{Opportunity:\} Bridge between scientific and metaphysical inquiry (cf. Aristotelian-Thomistic philosophy).\
\
\\subsection\{Getting Ahead: Strategic Recommendations\}\
\
\\subsubsection\{For Researchers\}\
\\begin\{itemize\}\
    \\item \\textbf\{Cosmologists:\} Test IAM predictions with upcoming Euclid, LSST, Roman Space Telescope data\
    \\item \\textbf\{Quantum physicists:\} Measure actualization signatures in quantum systems\
    \\item \\textbf\{Thermodynamicists:\} Explore actualization engine prototypes\
    \\item \\textbf\{Computer scientists:\} Develop actualization-based algorithms\
\\end\{itemize\}\
\
\\subsubsection\{For Industry\}\
\\begin\{itemize\}\
    \\item \\textbf\{Quantum computing companies:\} Hedge bets\'97invest in actualization computing R\\&D alongside traditional QC\
    \\item \\textbf\{Energy sector:\} Explore micro-scale energy harvesting via actualization dynamics\
    \\item \\textbf\{Materials science:\} Develop actualization-responsive materials\
\\end\{itemize\}\
\
\\subsubsection\{For Funding Agencies\}\
\\begin\{itemize\}\
    \\item Support cross-domain research bridging cosmology, quantum mechanics, and thermodynamics\
    \\item Fund experimental tests of actualization predictions\
    \\item Encourage independent researchers and human-AI collaborative research models\
\\end\{itemize\}\
\
\\section\{Methodology: Human-AI Collaborative Discovery\}\
\
\\subsection\{Development Process\}\
\
This work represents a novel research methodology:\
\
\\textbf\{Human contribution (H. Mahaffey):\}\
\\begin\{itemize\}\
    \\item Cross-domain pattern recognition (cosmology $\\leftrightarrow$ quantum mechanics $\\leftrightarrow$ thermodynamics)\
    \\item Philosophical grounding (Aristotelian-Thomistic metaphysics)\
    \\item Strategic vision (identifying implications and opportunities)\
    \\item Domain expertise integration (physics, philosophy, theology)\
\\end\{itemize\}\
\
\\textbf\{AI contribution (Claude, Anthropic):\}\
\\begin\{itemize\}\
    \\item Rapid synthesis of technical literature\
    \\item Mathematical formalization assistance\
    \\item Articulation and structuring of ideas\
    \\item Identification of testable predictions\
    \\item Pattern amplification and extension\
\\end\{itemize\}\
\
\\textbf\{Emergent capability:\}\
The partnership enables faster iteration, broader synthesis, and more rigorous formulation than either human or AI alone. This demonstrates a scalable model for exploratory theoretical research.\
\
\\subsection\{Transparency and Replicability\}\
\
This paper explicitly acknowledges AI collaboration to:\
\\begin\{itemize\}\
    \\item Maintain intellectual honesty\
    \\item Demonstrate new research methodology\
    \\item Encourage others to explore human-AI partnership\
    \\item Timestamp the collaborative process for future reference\
\\end\{itemize\}\
\
The core insights (actualization as fundamental, cross-scale applicability) originated from human pattern recognition; AI facilitated formalization, expansion, and articulation.\
\
\\section\{Conclusion and Future Work\}\
\
The Informational Actualization Model, if validated in cosmology, implies a far-reaching paradigm shift across physics, computation, and thermodynamics. The principle of irreversible actualization\'97the transition from potential to actual states\'97may be as fundamental as conservation laws or the second law of thermodynamics.\
\
\\subsection\{Key Takeaways\}\
\
\\begin\{enumerate\}\
    \\item \\textbf\{Quantum computing faces fundamental limits\} from actualization dynamics, but new paradigms (actualization computing) may emerge.\
    \\item \\textbf\{Thermodynamic devices\} exploiting actualization (heat engines, information batteries) become theoretically possible.\
    \\item \\textbf\{Foundational puzzles\} (measurement problem, arrow of time) find natural resolution in actualization framework.\
    \\item \\textbf\{Cross-domain unification\} spanning quantum mechanics, thermodynamics, and cosmology emerges from single principle.\
    \\item \\textbf\{Human-AI collaboration\} demonstrates new methodology for exploratory theoretical physics.\
\\end\{enumerate\}\
\
\\subsection\{Call for Collaboration\}\
\
This paper serves as an invitation:\
\
\\begin\{itemize\}\
    \\item \\textbf\{Experimentalists:\} Test actualization predictions in your domain\
    \\item \\textbf\{Theorists:\} Formalize and extend the framework\
    \\item \\textbf\{Engineers:\} Prototype actualization-based technologies\
    \\item \\textbf\{Philosophers:\} Explore metaphysical implications\
\\end\{itemize\}\
\
The author welcomes collaboration, particularly in:\
\\begin\{itemize\}\
    \\item Experimental validation of quantum computing predictions\
    \\item Development of actualization computing architectures\
    \\item Prototype thermodynamic actualization engines\
    \\item Cross-domain theoretical development\
\\end\{itemize\}\
\
\\subsection\{A Timestamp for the Future\}\
\
This work is published as a timestamp\'97a record of speculative ideas that may prove prescient if IAM cosmology is validated. Whether these predictions prove accurate or serve as instructive failures, documenting the exploratory process contributes to the scientific record.\
\
In the spirit of transparency: this is exploratory, not established science. But exploration is how paradigms shift.\
\
\\subsection\{Final Thought\}\
\
If actualization is fundamental, the universe is not a static collection of particles obeying timeless laws. It is an ongoing process\'97potential becoming actual, information being encoded, structure emerging from possibility.\
\
From quantum wavefunctions to cosmic expansion, the same principle applies:\
\
\\textit\{I AM\} $\\rightarrow$ potential $\\rightarrow$ actualization.\
\
Whether this vision proves correct depends on evidence yet to be gathered. But the pattern is too compelling not to explore.\
\
Let us see what the data reveals.\
\
\\vspace\{0.5cm\}\
\
\\noindent\\textit\{``The stone the builders rejected has become the cornerstone.''\} \\hfill ---Psalm 118:22\
\
\\begin\{thebibliography\}\{99\}\
\
\\bibitem\{Mahaffey2026\}\
H. W. Mahaffey,\
\\textit\{Holographic Black-Hole Cosmology: An Informational Resolution of the Hubble Tension\},\
OSF Preprints (2026), DOI: 10.17605/OSF.IO/KCZD9.\
Available at: \\url\{https://osf.io/kczd9\}\
\
\\bibitem\{Landauer1961\}\
R. Landauer,\
\\textit\{Irreversibility and Heat Generation in the Computing Process\},\
IBM J. Res. Dev. \\textbf\{5\}, 183 (1961).\
\
\\bibitem\{Planck2020\}\
Planck Collaboration,\
\\textit\{Planck 2018 results. VI. Cosmological parameters\},\
Astron. Astrophys. \\textbf\{641\}, A6 (2020), arXiv:1807.06209.\
\
\\bibitem\{Riess2022\}\
A. G. Riess et al.,\
\\textit\{A Comprehensive Measurement of the Local Value of the Hubble Constant with 1 km/s/Mpc Uncertainty from the Hubble Space Telescope and the SH0ES Team\},\
Astrophys. J. Lett. \\textbf\{934\}, L7 (2022), arXiv:2112.04510.\
\
\\bibitem\{DESI2024\}\
DESI Collaboration,\
\\textit\{DESI 2024 VI: Cosmological Constraints from the Measurements of Baryon Acoustic Oscillations\},\
arXiv:2404.03002 (2024).\
\
\\bibitem\{Bennett2003\}\
C. H. Bennett,\
\\textit\{Notes on Landauer's principle, reversible computation, and Maxwell's Demon\},\
Stud. Hist. Philos. Mod. Phys. \\textbf\{34\}, 501 (2003).\
\
\\bibitem\{Zurek2003\}\
W. H. Zurek,\
\\textit\{Decoherence, einselection, and the quantum origins of the classical\},\
Rev. Mod. Phys. \\textbf\{75\}, 715 (2003).\
\
\\bibitem\{Preskill2018\}\
J. Preskill,\
\\textit\{Quantum Computing in the NISQ era and beyond\},\
Quantum \\textbf\{2\}, 79 (2018), arXiv:1801.00862.\
\
\\bibitem\{Lloyd2000\}\
S. Lloyd,\
\\textit\{Ultimate physical limits to computation\},\
Nature \\textbf\{406\}, 1047 (2000).\
\
\\end\{thebibliography\}\
\
\\vspace\{1cm\}\
\
\\noindent\\rule\{\\textwidth\}\{0.4pt\}\
\
\\noindent\\textbf\{Acknowledgments:\} This work was developed through collaborative human-AI partnership with Claude (Anthropic). The author thanks the AI system for assistance with formalization, articulation, and extension of cross-domain patterns identified through human insight. All code and data supporting the cosmological IAM framework are publicly available at \\url\{https://github.com/hmahaffeyges/IAM-Validation\}.\
\
\\noindent\\textbf\{Data Availability:\} Cosmological validation code and datasets are available at the GitHub repository listed above. Speculative predictions in this paper are theoretical; experimental data will be shared as collaborations develop.\
\
\\noindent\\textbf\{Conflicts of Interest:\} The author declares no financial conflicts of interest. Intellectual collaboration with AI is disclosed transparently above.\
\
\\end\{document\}}